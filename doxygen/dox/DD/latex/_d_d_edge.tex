The structure of a Network is defined by {\bfseries{vertices}} connected by {\bfseries{edges}}. For a Dislocation\+Network, a vertex is a Dislocation\+Node, while an edge is a Dislocation\+Segment.

The information necessary to construct all Dislocation\+Segment (s) for the simulation time step N are stored in the file E/\+E\+\_\+\+N.\+txt (or E/\+E\+\_\+\+N.\+bin if binary output is selected, see the \mbox{\hyperlink{_d_d_input}{Dislocation Dynamics input file}}). Each line in E/\+E\+\_\+\+N.\+txt contains the properties of a distinct Dislocation\+Segment. The format of each line is\+: \begin{DoxyVerb}sourceID sinkID Bx By Bz Nx Ny Nz sourceTfactor sinkTfactor ncID
\end{DoxyVerb}
 where
\begin{DoxyItemize}
\item source\+ID is the Static\+ID of the Dislocation\+Node from which the Dislocation\+Segment starts
\item sink\+ID is the Static\+ID of the Dislocation\+Node at which the Dislocation\+Segment ends
\item Bx By Bz are the Cartesian coordinates of the Burgers vector of the Dislocation\+Segment
\item Nx Ny Nz are the Cartesian coordinates of the Glide\+Plane normal of the Dislocation\+Segment
\item source\+Tfactor a multiplicative factor (+1 or -\/1)of the parametric tangent of the source Dislocation\+Node
\item sink\+Tfactor a multiplicative factor (+1 or -\/1)of the parametric tangent of the sink Dislocation\+Node
\item nc\+ID is the Static\+ID of the Network\+Component to which the Dislocation\+Segment belongs
\end{DoxyItemize}

{\bfseries{Note}} that, when writing the initial input file, you only need to assign source\+ID, sink\+ID, Bx, By, and Bz. All other quantities can be set to zero, because they are internally re-\/computed by the code.

\begin{DoxySeeAlso}{See also}
\mbox{\hyperlink{_d_d_vertex}{Dislocation Vertex file}} 
\end{DoxySeeAlso}
